\documentclass{article}
\usepackage[utf8]{inputenc}
\usepackage{amssymb,amsmath,amsthm}
\usepackage{graphicx}
\usepackage[dvipsnames]{xcolor}
\usepackage{pstricks,pst-plot}
\usepackage{enumitem}
\usepackage{wasysym}
\usepackage{lscape}
\usepackage[ruled,vlined]{algorithm2e}

\input xy
\xyoption{all}

\setlength{\textwidth}{6.5in}
\setlength{\oddsidemargin}{-0in}
\setlength{\evensidemargin}{-0in}
\setlength{\textheight}{9in}
\setlength{\topmargin}{-0.4in}
%\newcounter{Pagenum}
%\addtocounter{Pagenum}{1}
\usepackage{fancyhdr}
\usepackage{multicol}
\newcommand\espace{\vrule height 20pt width 0pt}
\newcommand{\ds}{\displaystyle}
\linespread{1.1}
\pagenumbering{gobble}
\usepackage{multirow}
\setlist[description]{font=\normalfont\itshape\space}
\renewcommand{\arraystretch}{1.4}

\pagenumbering{arabic}

\newtheorem{theorem}{Theorem}
\renewcommand{\qedsymbol}{ \xrightswishingghost{} }

\pagestyle{fancy} \lhead{Catherine Huang}\rhead{Summer@ICERM 2020}

\begin{document}
\section{The Spectral Theorem}

\subsection{A self-adjoint operator $T \in \mathcal{L}(V)$ is symmetric.}
\begin{proof}
Consider a orthonormal basis of V: $e_1, ..., e_n$. Since we're working in an orthonormal basis, 
\begin{align*}
Te_k &= <Te_k, e_1>e_1 + \dotsc + <Te_k, e_n>e_n 
% Te_j &= <Te_j, e_1>e_1 + \dotsc + <Te_j, e_n>e_n 
\end{align*}
The entry at the jth row and kth column of $[\mathcal{M}(T, (e_1, ..., e_n), (e_1, ..., e_n))]_{jk} = <Te_k, e_j> $. The entry at the kth row and the jth column of $[\mathcal{M}(T, (e_1, ..., e_n), (e_1, ..., e_n))]_{kj} = <Te_j,e_k> = <e_j, Te_k>$ Over $\mathbf{F = R}$, $<Te_k, e_j> = <e_j, Te_k>$, so the corresponding matrix representation is a symmetric matrix.
\end{proof}
\subsection{For self-adjoint operator $T \in \mathcal{L}(V)$, $b, c \in \mathbf{R}$, and $b^2 - 4c < 0$, $T^2 + bT + c$ is invertible.}
\begin{proof}
An operator is invertible if 0 is not an eigenvalue. For any nonzero vector $v \in V$, consider 
\begin{align*}
<(T^2 + bT + c)v, v> &= <T^2v, v> + b<Tv, v> + c<v,v> \\ 
&\geq ||Tv||^2 -  b||Tv|| ||v|| + c||v||^2 \\
&= (||Tv||-\frac{b}{2}||v||)^2 + (c - \frac{b^2}{4})||v||^2 \\
&> 0
\end{align*}
We can conclude that $T^2 + bT + c$ is invertible because there are no 0 eigenvalues.
\end{proof}
\subsection{A self-adjoint operator $T \in \mathcal{L}(V)$ always has a real eigenvalue.}
\begin{proof}
For a nonzero vector v and $dim(V) = n$, consider the set of vectors $v, Tv, T^2v, \dots, T^nv.$ Because there are $n+1$ elements, there exists constants $a_0, \dots, a_n$ not all zero such that $a_0v + a_1Tv + \dots + a_nT^nv = 0$. This can be factorized as $d(T^2 + b_1T + c_1)\dotsc(T^2 + b_MT + c_M)(T-\lambda_1)\dotsc(T-\lambda_N)v = 0$ where $b_i^2-4c_i < 0$ and $\lambda, b, c, d \in \mathbf{R}$. From section 1.2, we know the inverses for the quadratic function exists, so $(T-\lambda_1I)\dotsc(T-\lambda_NI)v = 0$. Since we assumed v was a nonzero vector, there is some $\lambda_i$ such that $(T-\lambda_iI)v = 0$, ie $\lambda_i$ is a real eigenvalue.
\end{proof}
\subsection{For a self-adjoint operator $T \in \mathcal{L}(V)$ and suppose U is invariant in T, then $T|_{U^{\bot}}$ is a self-adjoint operator.}
\begin{proof}
Suppose $u \in U$ and $v \in U^{\bot}$, then $<u, Tv> = <Tu, v> = 0$ where the last equality is true because $Tu \in U$ and $v \in U^{\bot}$. This shows that $U^{\bot}$ is an invariant subspace as well. \\[1em]
Since $U^{\bot}$ is an invariant subspace, we know that $T|_{U^{\bot}} = T$ when applied on elements in $U^{\bot}$. Suppose $u, v \in U^{\bot}$, then $<T|_{U^{\bot}}u, v> = <Tu, v> = <u, Tv> = <u, <T|_{U^{\bot}}v>$, so $T|_{U^{\bot}}$ is a self-adjoint operator.
\end{proof}

\subsection{$T \in \mathcal{L}(V)$ is a self-adjoint operator (symmetric matrix) iff it can be unitarily diagonalizable with real numbers.}
\begin{proof}
We proceed by inducting on the $dim(V)$. \\
$\underline{\text{Base Case}}$: $dim(V) = 1$ This is trivially true because a vector is orthogonal to itself and we know it corresponds to a real eigenvalue from Section 1.3. \\ 
$\underline{\text{Inductive Hypothesis}}$: Suppose for all dimensions less than $dim V > 1$, the self-adjoint operator can be unitarily diagonalizable with real numbers.\\
$\underline{\text{Inductive Step}}$: Section 1.3 tells us that T has a real eigenvalue $\lambda$. Let $u$ be the corresponding eigenvector and because $Tu = \lambda u$, $U = \text{span}(u)$ is an invariant subspace. Section 1.4 tells us that $T|_{U^{\bot}}$ is a self-adjoint operator with dimension $dim(V)-1$. We can apply the inductive hypothesis on $T|_{U^{\bot}}$, so the vectors making up the orthogonal diagonalization consists of $\{u\} \cup U^{\bot}$.
\end{proof}
\end{document}